\chapter{Backup and Recovery Methods}

\lflogo
\minitoc

\begin{lfbox}[Learning Objectives]

   By the end of this session, you should be able to:

   \begin{itemize}

      \item
      Identify and prioritize data that needs backup.
      \item
      Understand the difference between backing up and
      archiving data.
      \item
      ​Efficiently employ different kinds of backup and restore
      methods, depending on the situation.
      \item
      Use a variety of compression methods such as \textbf{xz}
      and \textbf{gzip}, and combine them efficiently with the
      \textbf{tar} archiving program.
      \item
      Use \textbf{dd} for raw data copying, and \textbf{rsync}
      for efficient backup and synchronization across multiple
      machines.
      \item
      Describe some of the most well-known backup programs.

   \end{itemize}

\end{lfbox}

\optionalfootnotetext

\clearpage

\section{Backup Basics}

\begin{frame}
   {Why Backups?}

   \begin{itemize}
      \item Data is valuable
      \item Hardware fails
      \item Software fails
      \item People make mistakes
      \item Malicious people can cause deliberate damage
      \item Unexplained events happen
      \item Rewinds can be useful
   \end{itemize}

\end{frame}

\cprotect\note{ Whether you are administering only one personal
   system or a network of many machines, system backups
   are very important

   On-disk data is an important work product and is
   therefore a commodity that we wish to
   protect. Re-creating lost data costs time and money.
   Some data may even be unique and there may be no way to
   re-create it.

   While storage media reliability has increased, so has
   drive capacity. Even if the failure rate per byte
   decreases, unpredictable failures still occur.  It may
   be pessimistic to say there are only two kinds of
   drives; those that have failed and those that will
   fail, but it is essentially true.  Using \textbf{RAID}
   helps but backups are still needed.

   No software is perfect.  Bugs may destroy or corrupt
   data.  Even stable programs long in use can have
   problems.

   Everyone has heard \textbf{OOPS!} (or something much
   worse) coming from the next cubicle (or from their own
   mouth) at one time or another. Sometimes just a simple
   typing error can cause large scale destruction of files
   and data.

   It could be the canonical disgruntled employee or an
   external hacker with a point to make.  Security
   concerns and backup capabilities are very strongly
   related.

   Files can just disappear without you knowing how, who,
   or even when it occurred.

   Sometimes restoring to an earlier \textbf{snapshot} of
   all or part of the system may be required.
}



\begin{frame}
   {What Do We Need to Backup?}

   \begin{itemize}
      \item Definitely:
      \begin{itemize}
         \item Business-related data.
         \item System configuration files.
         \item User files (usually under \filelink{/home}).
      \end{itemize}
      \item Maybe:
      \begin{itemize}
         \item Spooling directories (for printing, mail etc.)
         \item Logging files (found in \filelink{/var/log}, and
         elsewhere).
      \end{itemize}
      \item Probably not:
      \begin{itemize}
         \item Software that can easily be re-installed; on a
         well-managed system this should be almost everything.
         \item The \filelink{/tmp} directory
      \end{itemize}
      \item Definitely not:
      \begin{itemize}
         \item Pseudo filesystems such as \filelink{/proc},
         \filelink{/dev} and \filelink{/sys}, and swap
      \end{itemize}
   \end{itemize}

\end{frame}

\cprotect\note{

   Obviously, files essential to your organization require
   backup.  Configuration files may change frequently, and
   along with individual user's files, require backup as
   well.

   Logging files can be important if you have to
   investigate your system's history, which can be
   particularly important for detecting intrusions and
   other security violations.

   You do not have to back up anything that can easily be
   re-installed.  Also the swap partitions (or files) and
   \filelink{/proc} filesystems are generally not useful
   or necessary to backup since data in these areas is
   basically temporary (just like in the \filelink{/tmp}
   directory).

}
\section{Backup vs Archive}
\begin{frame}
   {Backup vs Archive}

   \begin{itemize}
      \item
      All backup media have a finite life time before becoming
      unreadable
      \item
      Conventional Estimates:
      \begin{itemize}
         \item Magnetic Tapes: 10-30 years
         \item CDs and /DVDs: 3-10 years
         \item Hard Disks: 2-5 years
      \end{itemize}
      \item Lifetime very sensitive to:
      \begin{itemize}
         \item Environmental conditions (temperature, humidity
         etc.)
         \item Quality of media
         \item Having working software that can read data on current
         operating systems and hardware
      \end{itemize}
      \item Lifetime is sufficient for backup; not for permanent
      digital archiving
   \end{itemize}
\end{frame}

\cprotect\note{

   For life times longer than the usual backup timescale,
   data can be preserved using multiple copies plus copying
   over to newer media from time to time.

   For very long times (i.e., many decades, centuries etc.)
   standard methods do not work easily as everything can go
   obsolete:
   \begin{itemize}
      \item Hardware
      \item Software and Document Format
      \item Media
   \end{itemize}
   None of the inexpensive digital formats can actually
   compete with paper and film for long periods (if they are
   properly stored and continuously cared for -- like wine.)

   This is a problem serious people think about and there
   should be good solutions available before all is lost.  }


\begin{frame}
   {Tape Drives}
   \begin{itemize}
      \item Relatively slow
      \item Permit only sequential access
      \item Not as common as they once were
      \item Rarely used for primary backup today
      \item Still sometimes useful for off-site storage and archiving
   \end{itemize}

\end{frame}

\cprotect\note{

   Tape drives are not as common as they used to be. They are
   relatively slow and permit only sequential access.  On any
   modern setup they are rarely used for primary backup.  They
   are sometimes used for off-site storage for archival
   purposes for long time references.  However, magnetic tape
   drives always have only a finite lifetime without physical
   degradation and loss of data.

   Modern tape drives are usually of the \textbf{LTO}
   (\textbf{L}inear \textbf{T}ape \textbf{O}pen) variety,
   whose first versions appeared in the late 1990s as an open
   standards alternative; early formats were mostly
   proprietary.  Early versions held up to 100 GB; newer
   versions can hold 2.5 TB or more in a cartridge of the same
   size.

   Day to day backups are usually done with some form of
   \textbf{NAS} (\textbf{N}etwork \textbf{A}ttached
   \textbf{S}torage) or with \textbf{cloud}-based solutions,
   making new tape-based installations less and less
   attractive.  However, they can still be found and system
   administrators may be required to deal with them.

   In what follows we will try not to focus on particular
   physical forms for the backup media, and will speak more
   abstractly.

}
\section{Backup Methods and Strategies}
\begin{frame}
   {Backup Methods}

   \begin{itemize}
      \item \textbf{Full}:

      Backup all files on the system.
      \item \textbf{Incremental}:

      Backup all files that have changed since the last
      incremental or full backup.
      \item \textbf{Differential}:

      Backup all files that have changed since the last full
      backup.
      \item \textbf{Multiple level incremental}:

      Backup all files that have changed since the previous
      backup at the same or a previous level.
      \item \textbf{User}:

      Backup only files in a specific user's directory.
   \end{itemize}

\end{frame}

\cprotect\note{

   Several different kinds of backup methods can be used, often
   in concert with each other as listed.

   One should never have all backups residing in the same physical
   location as the systems being protected.  Otherwise, fire or other
   physical damage could lead to a total loss.  In the past this usually
   meant physically transporting magnetic tapes to a secure location.
   Today this is more likely to mean transferring backup files over the
   Internet to alternative physical locations.  Obviously, this has to
   be done in a secure way using encryption and other security
   precautions as is appropriate.

}

\begin{frame}
   {Backup Strategies}

   One useful strategy involving tapes (you can easily
   substitute other media in the description):

   \begin{enumerate}
      \item Use tape 1 for a full backup on Friday.
      \item Use tapes 2-5 for incremental backups on
      Monday-Thursday.
      \item Use tape 6 for full backup on second Friday.
      \item Use tapes 2-5 for incremental backups on second
      Monday-Thursday.
      \item Do not overwrite tape 1 until completion of full
      backup on tape 6.
      \item After full backup to tape 6, move tape 1 to
      external location for disaster recovery.
      \item For next full backup (next Friday) exchange tape 1
      for tape 6.
   \end{enumerate}

\end{frame}

\cprotect\note{

   We should note that backup methods are useless without
   associated \textbf{restore} methods.  One has to take
   into account the robustness, clarity and ease of both
   directions when selecting strategies.

   A good rule of thumb is to have at least two weeks of
   backups available.

   The simplest backup scheme is to do a full backup of
   everything once, and then perform incremental backups
   of everything that subsequently changes.  While full
   backups can take a lot of time, restoring from
   incremental backups can be more difficult and time
   consuming. Thus, one can use a mix of both to optimize
   time and effort.

}


\begin{frame}
   {Some Backup Related Utilities}

   \begin{itemize}
      \item \textbf{cpio}
      \item \textbf{tar}
      \item \textbf{gzip, bzip2, xz}
      \item \textbf{dd}
      \item \textbf{rsync}
      \item \textbf{dump/restore}
      \item \textbf{mt}

   \end{itemize}

\end{frame}

\cprotect\note{

   \textbf{cpio} and \textbf{tar} create and extract
   \textbf{archives} of files.

   The archives are often compressed with \textbf{gzip},
   \textbf{bzip2}, or \textbf{xz}.  The archive file may
   be written to disk, magnetic tape, or any other device
   which can hold files.  Archives are very useful for
   transferring files from one filesystem or machine to
   another.

   \textbf{dd} is a powerful utility often used to
   transfer raw data between media.  It can be used to
   copy entire partitions or entire disks.

   \textbf{rsync} is a powerful utility that can synchronize
   directory subtrees or entire filesystems across a
   network, or between different filesystem locations on
   a local machine.

   \textbf{dump} and \textbf{restore} are ancient
   utilities which were designed specifically for backups.
   They read from the filesystem directly (which is more
   efficient).  However, they must be restored only on the
   same filesystem type that they came from.  There are
   newer alternatives.

   \textbf{mt} is used for querying and positioning tapes
   before performing backups and restores.

}


\section{tar}
\begin{frame}
   {Using tar for Backups}

   \begin{itemize}
      \item Create an archive: use \texttt{-c} or just \texttt{c}
      \begin{cmd}
$ tar cvf  /dev/st0 /root
$ tar -cvf /dev/st0 /root
      \end{cmd}
      \item Create with multi volume option: use \texttt{-M}
      \begin{cmd}
$ tar -cMf /dev/st0 /root
      \end{cmd}
      Will be prompted to put next tape in

      \item Verify files with compare option: use
      \texttt{-d} or \verb?--compare?
      \begin{cmd}
$ tar --compare --verbose --file /dev/st0
$ tar -dvf /dev/st0
      \end{cmd}
      \item
      Note each option has a short form (one letter with a
      \verb?-?) or a long form (with \verb?--?)
      \item \textbf{tar} is by default recursive
   \end{itemize}

\end{frame}

\cprotect\note{

When creating a \textbf{tar} archive, for each
directory given as an argument, all files and
subdirectories will be included in the archive.
When restoring it reconstitutes directories as
necessary.

It even has a \texttt{-{}-newer} option that lets you
do incremental backups.

The version of \textbf{tar} used in \textbf{Linux} can
also handle backups that do not fit on one tape or
whatever device you use.

You can specify a device or file with the \texttt{-f}  or
\texttt{--file} options.

After you make a backup, you can make sure that it is
complete and correct using the verification option.

By default \textbf{tar} will recursively include all
subdirectories in the archive.

When you create an archive, \textbf{tar} prints a
message about removing leading slashes from the
absolute path name.  While this allows you to restore
the files anywhere, the default behavior can be
modified.

Most \textbf{tar} options can be given in short form
with one dash, or long form with two: \texttt{-c} is
completely equivalent to \texttt{--create}.

Also note that you can combine options (when using the
short notation) so that you do not have to type every
dash.

Furthermore, single-dashed \textbf{tar} options can be
used with or without dashes; i.e.,
\texttt{tar cvf file.tar dir1}  has the same result as
\texttt{tar -cvf file.tar dir1}.

}

\begin{frame}
   {Using tar for Restoring Files}

   \begin{itemize}
      \item Extract from an archive: use \texttt{-x} or
      \verb?--extract?
      \begin{cmd}
$ tar --extract --same-permissions --verbose --file /dev/st0
$ tar -xpvf /dev/st0
$ tar  xpvf /dev/st0
      \end{cmd}
      \item You may name specific files to restore
      \begin{cmd}
$ tar -xvf /dev/st0 somefile
      \end{cmd}
      \item Listing the contents of a \texttt{tar} backup
      \begin{cmd}
$ tar --list --file /dev/st0
$ tar -tf /dev/st0
      \end{cmd}
   \end{itemize}

\end{frame}

\cprotect\note{

   The \texttt{-x} or \texttt{--extract} option extracts files
   from an archive, all by default.  One can narrow the
   file extraction list by specifying only particular
   files.  If a directory is specified, all included files
   and subdirectories are also extracted.

   The \texttt{-p} or \texttt{--same-permissions} option
   ensures files are restored with their original
   permissions.

   The \texttt{-t} or \texttt{--list} option lists, but
   does not extract, the files in the archive.

}

\begin{frame}
   {Incremental Backups with tar}

   \begin{itemize}
      \item Use \verb?--newer? or
      \verb?--after-date? option
      \item Based on date
      \item Create a backup of all files modified after a
      certain date and compress them:
      \begin{cmd}
$ tar --create --newer '2011-12-1' -vzf backup1.tgz /var/tmp
$ tar --create --after-date '2011-12-1' -vzf backup1.tgz /var/tmp
      \end{cmd}
      Either form creates a backup archive of all files in
      \filelink{/var/tmp} which were modified after December 1, 2011.
   \end{itemize}

\end{frame}

\cprotect\note{

   You can do an incremental backup with \textbf{tar}
   using the \texttt{-N} (or the equivalent
   \texttt{--newer}), or the \texttt{--after-date}
   options.  Either option requires specifying either a
   date or a qualified (reference) file name.

   Because \textbf{tar} only looks at a file's date, it
   does not consider any other changes to the file, such
   as permissions or file name.  To include files with
   these changes in the incremental backup, use
   \textbf{find} and create a list of files to be backed
   up.

   \textbf{Note}: When followed by an option like
   \texttt{--newer} you must use the dash in options
   like \texttt{-vzf}, or \textbf{tar} will get confused.
   This kind of option specification confusion sometimes
   occurs with old \textbf{UNIX} utilities like
   \textbf{ps} and \textbf{tar} with complicated histories
   involving different families of \textbf{UNIX}.

}




\section{Compression: gzip, bzip2 and xz and Backups}

\begin{frame}
   {Archive Compression Methods}

   \begin{itemize}
      \item \textbf{Compressing} files saves
      disk space and/or network transmission time.
      \item
      In order of increasing compression efficiency
      which comes at the cost of longer compression times:
      \begin{itemize}
         \item \textbf{gzip}: Uses Lempel-Ziv Coding (LZ77), and
         produces \texttt{.gz} files
         \item \textbf{bzip2}: Uses Burrows-Wheeler block sorting
         text compression algorithm and Huffman coding, and
         produces \texttt{.bz2} files
         \item \textbf{xz}: Produces \texttt{.xz} files. and also
         supports legacy \texttt{.lzma} format

      \end{itemize}
      \item
      Modern machines will often find the
      \textbf{compress $\rightarrow$ transmit $\rightarrow$ decompress} cycle
      faster than just transmitting (or copying) an
      uncompressed file.
      \item
      \url{https://www.kernel.org} only uses \textbf{xz} format now
      for downloading \textbf{Linux} kernels
   \end{itemize}

\end{frame}

\cprotect\note{

   The compression utilities are very easily (and often) used in
   combination with \textbf{tar}:
   \begin{cmd}
$ tar zcvf source.tar.gz  source
$ tar jcvf source.tar.bz2 source
$ tar Jcvf source.tar.xz  source
   \end{cmd}
   for producing a compressed archive.  Note the first command has the
   exact same effect as:
   \begin{cmd}
$ tar cvf source.tar source ; gzip -v source.tar
   \end{cmd}
   but is more efficient because there is no intermediate file storage,
   and archiving and compression happen simultaneously in the pipeline.

   For decompression:
   \begin{cmd}
$ tar xzvf source.tar.gz
$ tar xjvf source.tar.bz2
$ tar xJvf source.tar.xz
   \end{cmd}
   or even simpler:
   \begin{cmd}
$ tar xvf source.tar.gz
   \end{cmd}
   as modern versions of \textbf{tar} can sense the method of
   compression and take care of it automatically.

   Obviously it is not worth using these methods on archives
   whose component files are already compressed, such as
   \texttt{} images, or \texttt{.pdf} files etc.

}



\section{dd}

\begin{frame}
   {dd}

   \begin{itemize}
      \item Used to copy raw data
      \item Low level copy of data
      \item Syntax: \texttt{dd if=input-file of=output-file
         options}
      \begin{cmd}
$ dd if=/dev/sda of=/dev/sdb
      \end{cmd}
      \item Can be used to create files
      \begin{cmd}
$ dd if=/dev/zero of=file1 bs=1M count=10
      \end{cmd}
      \item Can backup entire hard drives or partitions
      \item Can copy CD or DVDs
   \end{itemize}

\end{frame}

\cprotect\note{

   \textbf{dd} is a common \textbf{UNIX}-based program whose primary
   purpose is the low-level copying and conversion of raw
   data. It is used to copy a specified number of bytes
   or blocks, performing on-the-fly byte order
   conversions, as well as being able to convert data
   from one form to another. It can also be used to copy
   regions of raw device files, for example backing up
   the boot sector of a hard disk, or to read fixed
   amounts of data from special files like
   \filelink{/dev/zero} or \filelink{/dev/random}.

   Backup an entire hard drive to another.

   \begin{cmd}
$ dd if=/dev/sda of=/dev/sdb
   \end{cmd}

   Create an image of a hard disk.

   \begin{cmd}
$ dd if=/dev/sda of=sdadisk.img
   \end{cmd}

   Backup a partition.

   \begin{cmd}
$ dd if=/dev/sda1 of=partition1.img
   \end{cmd}

   Backup a CD ROM.

   \begin{cmd}
$ dd if=/dev/cdrom of=tgsservice.iso bs=2048
   \end{cmd}
}




\section{rsync}
\begin{frame}
   {Using rsync for Backups}

   \begin{itemize}
      \item
      Usage:
      \begin{cmd}
$ rsync [options] sourcefile destinationfile
      \end{cmd}
      \item
      Between local machine and a network target:
      \begin{cmd}
$ rsync file.tar someone@backup.mydomain:/usr/local
      \end{cmd}
      \item
      Test before doing with \verb?--dry-run?:
      \begin{cmd}
$ rsync -r --dry-run /usr/local /BACKUP/usr
      \end{cmd}
   \end{itemize}

\end{frame}

\cprotect\note{

   \textbf{rsync} (\textbf{r}emote \textbf{sync}hronize)
   is used to transfer files across a network (or between
   different locations on the same machine).

   The source and destination can take the form of
   \texttt{target:path} where \texttt{target} can be in
   the form of \texttt{[user@]host}.  The \texttt{user@}
   part is optional and used if the remote user is
   different from the local user.  Thus, these are all
   possible \textbf{rsync} commands:
   \begin{cmd}
$ rsync file.tar someone@backup.mydomain:/usr/local
$ rsync -r --dry-run /usr/local /BACKUP/usr
   \end{cmd}
   You have be very careful with \textbf{rsync} about
   exact location specifications (especially if you use
   the \verb?--delete? option), so it is highly
   recommended to use the \verb?--dry-run? option
   first, and then repeat the command if the projected
   action looks correct.

   \textbf{rsync} is very clever; it checks local files
   against remote files in small chunks, and it is very
   efficient in that when copying one directory to a
   similar directory, only the differences are copied over
   the network
   with the first directory.  One often uses the
   \texttt{-r} option which causes \textbf{rsync} to
   recursively walk down the directory tree copying all
   files and directories below the one listed as the
   \texttt{sourcefile}.  Thus, a very useful way to back
   up a project directory might be similar to:

   \begin{cmd}
$ rsync -r project-X archive-machine:archives/project-X
   \end{cmd}

   A simple (and very effective and very fast) backup
   strategy is to simply duplicate directories or
   partitions across a network with \textbf{rsync}
   commands and to do so frequently.

}


\section{Backup Programs **}
\begin{frame}
   {Backup Programs}
   \begin{itemize}
      \item \textbf{Amanda}
      \item \textbf{Bacula}
      \item \textbf{Clonezilla}
   \end{itemize}

\end{frame}

\cprotect\note{

   There is no shortage of available backup program suites
   available for \textbf{Linux}, including proprietary
   applications or those supplied by storage vendors, as
   well as open-source applications.  Several that are
   particularly well known are:

   \begin{itemize}
      \item \textbf{Amanda} (\textbf{A}dvanced
      \textbf{M}aryland \textbf{A}utomatic \textbf{N}etwork
      \textbf{D}isk \textbf{A}rchiver) uses native utilities
      (including \textbf{tar} and \textbf{dump}) but is far
      more robust and controllable.

      \textbf{Amanda} is generally available on Enterprise
      \textbf{Linux} systems through the usual repositories.
      Complete information can be found at
      \url{http://www.amanda.org}.

      \item
      \textbf{Bacula} is designed for automatic backup on
      heterogeneous networks.  It can be rather complicated
      to use and is recommended (by its authors) only to
      experienced administrators.

      \textbf{Bacula} is generally available on Enterprise
      \textbf{Linux} systems through the usual repositories.
      Complete information can be found at
      \url{https://www.bacula.org/7.0.x-manuals/en/main/Main_Reference.html}.
      \item
      \textbf{Clonezilla} is a very robust \textbf{disk
         cloning} program, which can make images of disks and
      deploy them, either to restore a backup, or to be used
      for \textbf{ghosting}, to provide an image that can be
      used to install many machines.  Complete information
      can be found at \url{https://clonezilla.org}.

      The program comes in two versions: \textbf{Clonezilla
         Live}, which is good for single machine backup and
      recovery, and \textbf{Clonezilla SE, server edition},
      which can clone to many computers at the same time.

      \textbf{Clonezilla} is not very hard to use and is
      extremely flexible, supporting many operating systems
      (not just \textbf{Linux}), filesystem types, and boot
      loaders.

   \end{itemize}

}



\clearpage\section{Labs}\begin{Lab}
   \begin{exe} {Using tar for Backup}

      \begin{enumerate}
         \item
         Create a directory called \verb?backup? and in it
         place a compressed \textbf{tar} archive of all the
         files under \filelink{/usr/include}, with the highest
         level directory being \verb?include?.  You can use
         any compression method (\textbf{gzip}, \textbf{bzip2}
         or \textbf{xzip}).
         \item
         List the files in the archive.
         \item
         Create a directory called \verb?restore? and
         unpack and decompress the archive.
         \item
         Compare the
         contents with the original directory the archive
         was made from.
      \end{enumerate}

      \begin{sol}

         \begin{enumerate}
            \item
            \begin{cmd}
$ mkdir /tmp/backup
$ cd /usr ; tar zcvf /tmp/backup/include.tar.gz include
$ cd /usr ; tar jcvf /tmp/backup/include.tar.bz2 include
$ cd /usr ; tar Jcvf /tmp/backup/include.tar.xz include
         \end{cmd}
            or
            \begin{cmd}
$ tar -C /usr -zcf include.tar.gz include
$ tar -C /usr -jcf include.tar.bz2 include
$ tar -C /usr -Jcf include.tar.xz include
         \end{cmd}
            Notice the efficacy of the compression between the three
            methods:
            \begin{cmd}
$ du -sh /usr/include
         \end{cmd}

            \begin{out}[]
55M	/usr/include
         \end{out}
            \item
            \begin{cmd}
$ ls -lh include.tar.*
         \end{cmd}

            \begin{out}[]
c8:/tmp/backup>ls -lh
total 17M
-rw-rw-r-- 1 coop coop 5.3M Jul 18 08:17 include.tar.bz2
-rw-rw-r-- 1 coop coop 6.7M Jul 18 08:16 include.tar.gz
-rw-rw-r-- 1 coop coop 4.5M Jul 18 08:18 include.tar.xz
         \end{out}
            \item
            \begin{cmd}
$ tar tvf include.tar.xz
         \end{cmd}

            \begin{out}[]
qdrwxr-xr-x root/root         0 2014-10-29 07:04 include/
-rw-r--r-- root/root     42780 2014-08-26 12:24 include/unistd.h
-rw-r--r-- root/root       957 2014-08-26 12:24 include/re_comp.h
-rw-r--r-- root/root     22096 2014-08-26 12:24 include/regex.h
-rw-r--r-- root/root      7154 2014-08-26 12:25 include/link.h
.....
         \end{out}
            Note it is not necessary to give the \verb?j?,
            \verb?J?, or \verb?z?  option when decompressing;
            \textbf{tar} is smart enough to figure out what is
            needed.
            \item
            \begin{cmd}
$ cd .. ; mkdir restore ; cd restore
$ tar xvf ../backup/include.tar.bz2
         \end{cmd}

            \begin{out}[]
include/
include/unistd.h
include/re_comp.h
include/regex.h
include/link
.....
$ diff -qr include /usr/include
         \end{out}
         \end{enumerate}
      \end{sol}
   \end{exe}

   \begin{exe} {Using rsync for Backup}

      \begin{enumerate}
         \item
         Using \textbf{rsync}, we will again create a complete
         copy of \filelink{/usr/include} in your backup
         directory:
         \begin{cmd}
$ rm -rf include
$ rsync -av /usr/include .
      \end{cmd}

         \begin{out}[]
sending incremental file list
include/
include/FlexLexer.h
include/_G_config.h
include/a.out.h
include/aio.h
.....
      \end{out}
         \item
         Let's run the command a second time and see if it
         does anything:
         \begin{cmd}
$ rsync -av /usr/include .
      \end{cmd}

         \begin{out}[]
sending incremental file list

sent 127398 bytes  received 188 bytes  255172.00 bytes/sec
total size is 41239979  speedup is 323.23
      \end{out}
         \item
         One confusing thing about \textbf{rsync} is you might
         have expected the right command to be:
         \begin{cmd}
$ rsync -av /usr/include include
      \end{cmd}

         \begin{out}[]
sending incremental file list
...
      \end{out}
         However, if you do this, you'll find it actually
         creates a new directory, \filelink{include/include}!
         \item
         To get rid of the extra files you can use the
         \verb?--delete? option:
         \begin{cmd}
$ rsync -av --delete /usr/include .
      \end{cmd}

         \begin{out}[]
sending incremental file list
include/
deleting include/include/xen/privcmd.h
deleting include/include/xen/evtchn.h
....
deleting include/include/FlexLexer.h
deleting include/include/

sent 127401 bytes  received 191 bytes  85061.33 bytes/sec
total size is 41239979  speedup is 323.22
      \end{out}

         \item For another simple exercise,
         remove a subdirectory tree in your backup copy and
         then run \textbf{rsync} again with and without the
         \verb?--dry-run? option:
         \begin{cmd}
$ rm -rf include/xen
$ rsync -av --delete --dry-run /usr/include .
      \end{cmd}

         \begin{out}[]
sending incremental file list
include/
include/xen/
include/xen/evtchn.h
include/xen/privcmd.h

sent 127412 bytes  received 202 bytes  255228.00 bytes/sec
total size is 41239979  speedup is 323.16 (DRY RUN)
      \end{out}
         \begin{cmd}
$ rsync -av --delete  /usr/include .
      \end{cmd}
         \item  A simple script with a good set of options for using
         \textbf{rsync}:
         \begin{shlst}[script using \textbf{rsync}]
         #!/bin/sh
         set -x

         rsync --progress -avrxH --delete $*
      \end{shlst}
         which will work on a local machine as well as over
         the network.  Note the important \verb?-x? option
         which stops \textbf{rsync} from crossing
         filesystem boundaries.
      \end{enumerate}
      \begin{lfbox}[Extra Credit]
         For more fun, if you have access to more than one computer,
         try doing these steps with source and destination on different
         machines.
      \end{lfbox}
   \end{exe}

\end{Lab}


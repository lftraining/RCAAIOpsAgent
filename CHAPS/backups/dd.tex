\section{dd}

\begin{frame}
   {dd}

   \begin{itemize}
      \item Used to copy raw data
      \item Low level copy of data
      \item Syntax: \texttt{dd if=input-file of=output-file
         options}
      \begin{cmd}
$ dd if=/dev/sda of=/dev/sdb
      \end{cmd}
      \item Can be used to create files
      \begin{cmd}
$ dd if=/dev/zero of=file1 bs=1M count=10
      \end{cmd}
      \item Can backup entire hard drives or partitions
      \item Can copy CD or DVDs
   \end{itemize}

\end{frame}

\cprotect\note{

   \textbf{dd} is a common \textbf{UNIX}-based program whose primary
   purpose is the low-level copying and conversion of raw
   data. It is used to copy a specified number of bytes
   or blocks, performing on-the-fly byte order
   conversions, as well as being able to convert data
   from one form to another. It can also be used to copy
   regions of raw device files, for example backing up
   the boot sector of a hard disk, or to read fixed
   amounts of data from special files like
   \filelink{/dev/zero} or \filelink{/dev/random}.

   Backup an entire hard drive to another.

   \begin{cmd}
$ dd if=/dev/sda of=/dev/sdb
   \end{cmd}

   Create an image of a hard disk.

   \begin{cmd}
$ dd if=/dev/sda of=sdadisk.img
   \end{cmd}

   Backup a partition.

   \begin{cmd}
$ dd if=/dev/sda1 of=partition1.img
   \end{cmd}

   Backup a CD ROM.

   \begin{cmd}
$ dd if=/dev/cdrom of=tgsservice.iso bs=2048
   \end{cmd}
}




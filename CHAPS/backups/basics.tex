\section{Backup Basics}

\begin{frame}
   {Why Backups?}

   \begin{itemize}
      \item Data is valuable
      \item Hardware fails
      \item Software fails
      \item People make mistakes
      \item Malicious people can cause deliberate damage
      \item Unexplained events happen
      \item Rewinds can be useful
   \end{itemize}

\end{frame}

\cprotect\note{ Whether you are administering only one personal
   system or a network of many machines, system backups
   are very important

   On-disk data is an important work product and is
   therefore a commodity that we wish to
   protect. Re-creating lost data costs time and money.
   Some data may even be unique and there may be no way to
   re-create it.

   While storage media reliability has increased, so has
   drive capacity. Even if the failure rate per byte
   decreases, unpredictable failures still occur.  It may
   be pessimistic to say there are only two kinds of
   drives; those that have failed and those that will
   fail, but it is essentially true.  Using \textbf{RAID}
   helps but backups are still needed.

   No software is perfect.  Bugs may destroy or corrupt
   data.  Even stable programs long in use can have
   problems.

   Everyone has heard \textbf{OOPS!} (or something much
   worse) coming from the next cubicle (or from their own
   mouth) at one time or another. Sometimes just a simple
   typing error can cause large scale destruction of files
   and data.

   It could be the canonical disgruntled employee or an
   external hacker with a point to make.  Security
   concerns and backup capabilities are very strongly
   related.

   Files can just disappear without you knowing how, who,
   or even when it occurred.

   Sometimes restoring to an earlier \textbf{snapshot} of
   all or part of the system may be required.
}



\begin{frame}
   {What Do We Need to Backup?}

   \begin{itemize}
      \item Definitely:
      \begin{itemize}
         \item Business-related data.
         \item System configuration files.
         \item User files (usually under \filelink{/home}).
      \end{itemize}
      \item Maybe:
      \begin{itemize}
         \item Spooling directories (for printing, mail etc.)
         \item Logging files (found in \filelink{/var/log}, and
         elsewhere).
      \end{itemize}
      \item Probably not:
      \begin{itemize}
         \item Software that can easily be re-installed; on a
         well-managed system this should be almost everything.
         \item The \filelink{/tmp} directory
      \end{itemize}
      \item Definitely not:
      \begin{itemize}
         \item Pseudo filesystems such as \filelink{/proc},
         \filelink{/dev} and \filelink{/sys}, and swap
      \end{itemize}
   \end{itemize}

\end{frame}

\cprotect\note{

   Obviously, files essential to your organization require
   backup.  Configuration files may change frequently, and
   along with individual user's files, require backup as
   well.

   Logging files can be important if you have to
   investigate your system's history, which can be
   particularly important for detecting intrusions and
   other security violations.

   You do not have to back up anything that can easily be
   re-installed.  Also the swap partitions (or files) and
   \filelink{/proc} filesystems are generally not useful
   or necessary to backup since data in these areas is
   basically temporary (just like in the \filelink{/tmp}
   directory).

}
\section{Backup vs Archive}
\begin{frame}
   {Backup vs Archive}

   \begin{itemize}
      \item
      All backup media have a finite life time before becoming
      unreadable
      \item
      Conventional Estimates:
      \begin{itemize}
         \item Magnetic Tapes: 10-30 years
         \item CDs and /DVDs: 3-10 years
         \item Hard Disks: 2-5 years
      \end{itemize}
      \item Lifetime very sensitive to:
      \begin{itemize}
         \item Environmental conditions (temperature, humidity
         etc.)
         \item Quality of media
         \item Having working software that can read data on current
         operating systems and hardware
      \end{itemize}
      \item Lifetime is sufficient for backup; not for permanent
      digital archiving
   \end{itemize}
\end{frame}

\cprotect\note{

   For life times longer than the usual backup timescale,
   data can be preserved using multiple copies plus copying
   over to newer media from time to time.

   For very long times (i.e., many decades, centuries etc.)
   standard methods do not work easily as everything can go
   obsolete:
   \begin{itemize}
      \item Hardware
      \item Software and Document Format
      \item Media
   \end{itemize}
   None of the inexpensive digital formats can actually
   compete with paper and film for long periods (if they are
   properly stored and continuously cared for -- like wine.)

   This is a problem serious people think about and there
   should be good solutions available before all is lost.  }


\begin{frame}
   {Tape Drives}
   \begin{itemize}
      \item Relatively slow
      \item Permit only sequential access
      \item Not as common as they once were
      \item Rarely used for primary backup today
      \item Still sometimes useful for off-site storage and archiving
   \end{itemize}

\end{frame}

\cprotect\note{

   Tape drives are not as common as they used to be. They are
   relatively slow and permit only sequential access.  On any
   modern setup they are rarely used for primary backup.  They
   are sometimes used for off-site storage for archival
   purposes for long time references.  However, magnetic tape
   drives always have only a finite lifetime without physical
   degradation and loss of data.

   Modern tape drives are usually of the \textbf{LTO}
   (\textbf{L}inear \textbf{T}ape \textbf{O}pen) variety,
   whose first versions appeared in the late 1990s as an open
   standards alternative; early formats were mostly
   proprietary.  Early versions held up to 100 GB; newer
   versions can hold 2.5 TB or more in a cartridge of the same
   size.

   Day to day backups are usually done with some form of
   \textbf{NAS} (\textbf{N}etwork \textbf{A}ttached
   \textbf{S}torage) or with \textbf{cloud}-based solutions,
   making new tape-based installations less and less
   attractive.  However, they can still be found and system
   administrators may be required to deal with them.

   In what follows we will try not to focus on particular
   physical forms for the backup media, and will speak more
   abstractly.

}
\section{Backup Methods and Strategies}
\begin{frame}
   {Backup Methods}

   \begin{itemize}
      \item \textbf{Full}:

      Backup all files on the system.
      \item \textbf{Incremental}:

      Backup all files that have changed since the last
      incremental or full backup.
      \item \textbf{Differential}:

      Backup all files that have changed since the last full
      backup.
      \item \textbf{Multiple level incremental}:

      Backup all files that have changed since the previous
      backup at the same or a previous level.
      \item \textbf{User}:

      Backup only files in a specific user's directory.
   \end{itemize}

\end{frame}

\cprotect\note{

   Several different kinds of backup methods can be used, often
   in concert with each other as listed.

   One should never have all backups residing in the same physical
   location as the systems being protected.  Otherwise, fire or other
   physical damage could lead to a total loss.  In the past this usually
   meant physically transporting magnetic tapes to a secure location.
   Today this is more likely to mean transferring backup files over the
   Internet to alternative physical locations.  Obviously, this has to
   be done in a secure way using encryption and other security
   precautions as is appropriate.

}

\begin{frame}
   {Backup Strategies}

   One useful strategy involving tapes (you can easily
   substitute other media in the description):

   \begin{enumerate}
      \item Use tape 1 for a full backup on Friday.
      \item Use tapes 2-5 for incremental backups on
      Monday-Thursday.
      \item Use tape 6 for full backup on second Friday.
      \item Use tapes 2-5 for incremental backups on second
      Monday-Thursday.
      \item Do not overwrite tape 1 until completion of full
      backup on tape 6.
      \item After full backup to tape 6, move tape 1 to
      external location for disaster recovery.
      \item For next full backup (next Friday) exchange tape 1
      for tape 6.
   \end{enumerate}

\end{frame}

\cprotect\note{

   We should note that backup methods are useless without
   associated \textbf{restore} methods.  One has to take
   into account the robustness, clarity and ease of both
   directions when selecting strategies.

   A good rule of thumb is to have at least two weeks of
   backups available.

   The simplest backup scheme is to do a full backup of
   everything once, and then perform incremental backups
   of everything that subsequently changes.  While full
   backups can take a lot of time, restoring from
   incremental backups can be more difficult and time
   consuming. Thus, one can use a mix of both to optimize
   time and effort.

}


\begin{frame}
   {Some Backup Related Utilities}

   \begin{itemize}
      \item \textbf{cpio}
      \item \textbf{tar}
      \item \textbf{gzip, bzip2, xz}
      \item \textbf{dd}
      \item \textbf{rsync}
      \item \textbf{dump/restore}
      \item \textbf{mt}

   \end{itemize}

\end{frame}

\cprotect\note{

   \textbf{cpio} and \textbf{tar} create and extract
   \textbf{archives} of files.

   The archives are often compressed with \textbf{gzip},
   \textbf{bzip2}, or \textbf{xz}.  The archive file may
   be written to disk, magnetic tape, or any other device
   which can hold files.  Archives are very useful for
   transferring files from one filesystem or machine to
   another.

   \textbf{dd} is a powerful utility often used to
   transfer raw data between media.  It can be used to
   copy entire partitions or entire disks.

   \textbf{rsync} is a powerful utility that can synchronize
   directory subtrees or entire filesystems across a
   network, or between different filesystem locations on
   a local machine.

   \textbf{dump} and \textbf{restore} are ancient
   utilities which were designed specifically for backups.
   They read from the filesystem directly (which is more
   efficient).  However, they must be restored only on the
   same filesystem type that they came from.  There are
   newer alternatives.

   \textbf{mt} is used for querying and positioning tapes
   before performing backups and restores.

}


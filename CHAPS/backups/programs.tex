\section{Backup Programs **}
\begin{frame}
   {Backup Programs}
   \begin{itemize}
      \item \textbf{Amanda}
      \item \textbf{Bacula}
      \item \textbf{Clonezilla}
   \end{itemize}

\end{frame}

\cprotect\note{

   There is no shortage of available backup program suites
   available for \textbf{Linux}, including proprietary
   applications or those supplied by storage vendors, as
   well as open-source applications.  Several that are
   particularly well known are:

   \begin{itemize}
      \item \textbf{Amanda} (\textbf{A}dvanced
      \textbf{M}aryland \textbf{A}utomatic \textbf{N}etwork
      \textbf{D}isk \textbf{A}rchiver) uses native utilities
      (including \textbf{tar} and \textbf{dump}) but is far
      more robust and controllable.

      \textbf{Amanda} is generally available on Enterprise
      \textbf{Linux} systems through the usual repositories.
      Complete information can be found at
      \url{http://www.amanda.org}.

      \item
      \textbf{Bacula} is designed for automatic backup on
      heterogeneous networks.  It can be rather complicated
      to use and is recommended (by its authors) only to
      experienced administrators.

      \textbf{Bacula} is generally available on Enterprise
      \textbf{Linux} systems through the usual repositories.
      Complete information can be found at
      \url{https://www.bacula.org/7.0.x-manuals/en/main/Main_Reference.html}.
      \item
      \textbf{Clonezilla} is a very robust \textbf{disk
         cloning} program, which can make images of disks and
      deploy them, either to restore a backup, or to be used
      for \textbf{ghosting}, to provide an image that can be
      used to install many machines.  Complete information
      can be found at \url{https://clonezilla.org}.

      The program comes in two versions: \textbf{Clonezilla
         Live}, which is good for single machine backup and
      recovery, and \textbf{Clonezilla SE, server edition},
      which can clone to many computers at the same time.

      \textbf{Clonezilla} is not very hard to use and is
      extremely flexible, supporting many operating systems
      (not just \textbf{Linux}), filesystem types, and boot
      loaders.

   \end{itemize}

}


\clearpage
\section{/bin}
\begin{frame}
   {/bin}

   \begin{itemize}
      \item Contains executable programs and scripts needed by
      both system administrators and unprivileged users, which
      are required when no other filesystems have yet been
      mounted, for example when booting into \textbf{single
         user} or recovery mode.
      \item May also contain executables which are used
      indirectly by scripts.
      \item May not include any subdirectories.


      \begin{lfbox}[\texttt{/bin} and \texttt{/usr/bin}]
         \normalsize{ Some recent distributions have abandoned
            the strategy of separating \filelink{/bin} and
            \filelink{/usr/bin} (as well as \filelink{/sbin} and
            \filelink{/usr/sbin}) and just have one directory
            with symbolic links, thereby preserving a two
            directory view.  They view the time-honored concept
            of enabling the possibility of placing
            \filelink{/usr} on a separate partition to be
            mounted after boot as obsolete.  }
      \end{lfbox}
   \end{itemize}
\end{frame}

\cprotect\note{

Required programs which must exist in \filelink{/bin/}
include:
\begin{quote}
   \textbf{cat, chgrp, chmod, chown, cp, date, dd, df,
      dmesg, echo, false, hostname, kill, ln, login, ls,
      mkdir, mknod, more, mount, mv, ps, pwd, rm, rmdir, sed,
      sh, stty, su, sync, true, umount} and \textbf{uname}.
\end{quote}
\textbf{[} and \textbf{test} may be there as well, and
optionally, it may include:
\begin{quote}
   \textbf{csh, ed, tar, cpio, gunzip, zcat, netstat} and
   \textbf{ping}.
\end{quote}
Command binaries that are deemed not essential enough to
merit a place in \filelink{/bin} go in \filelink{/usr/bin}.
Programs required only by non-root users are placed in this
category.

}
\begin{frame}
   {/bin Example}

   \begin{figure}[H]
      \includegraphics[width=6.5in]{IMAGES/lsbin}
      \caption{/bin Directory: Ubuntu 18.04}
   \end{figure}
\end{frame}

\cprotect\note {

   Recent distribution versions of \textbf{RHEL},
   \textbf{CentOS}, \textbf{Fedora}, and \textbf{Ubuntu}
   have symbolically linked \filelink{/bin} and
   \filelink{/usr/bin} so they are actually the same.

   The above screenshot is from \textbf{Ubuntu 18.04}; later
   versions no longer have a separate un-linked directory.

}


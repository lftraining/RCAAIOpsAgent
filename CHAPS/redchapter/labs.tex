\clearpage\section{Labs}\begin{Lab}

   \ifINSTRUCTORLED
      \begin{videobox}
         \begin{alltt}
\reslink{using_package_gui_ubuntu_demo.mp4} \end{alltt}
      \end{videobox}
   \fi

   \setcounter{labno}{-1}
   \begin{exe} {dpkg Required}

      \begin{debianfamilybox}
         To do these labs you need to have access to a system that
         is \textbf{Debian}-based, such as \textbf{Debian},
         \textbf{Ubuntu}, or \textbf{Linux Mint}.
      \end{debianfamilybox}
   \end{exe}

   \begin{exe} {Basic APT Commands}


      \begin{enumerate}
         \item
         Check to see if there are any available updates for
         your system.
         \item
         List all installed kernel-related packages, and list
         all installed or available ones.
         \item
         Install the \textbf{apache2-dev} package, or anything
         else you might not have installed yet.  Doing a simple:
         \begin{cmd}
$ apt-cache pkgnames
      \end{cmd}

         will let you see a complete list; you may want to give a wildcard
         argument to narrow the list.
      \end{enumerate}

      \begin{sol}

         \begin{enumerate}
            \item
            First synchronize the package index files with
            remote repositories:
            \begin{cmd}
$ sudo apt update
         \end{cmd}
            To actually upgrade:
            \begin{cmd}
$ sudo apt upgrade
$ sudo apt -u upgrade
         \end{cmd}
            (You can also use \verb?dist-upgrade? as
            discussed earlier.)  Only the first form will
            try to do the installations.
            \item
            \begin{cmd}
$ apt-cache search "kernel"
$ apt-cache search -n "kernel"
$ apt-cache pkgnames "kernel"
         \end{cmd}
            The second and third forms only find packages
            that have \verb?kernel?  in their name.
            \begin{cmd}
$ dpkg --get-selections "*kernel*"
         \end{cmd}
            to get only installed packages.  Note that on
            \textbf{Debian}-based systems you probably
            should use \verb?linux? not \verb?kernel? for
            kernel-related packages as they don't usually
            have \verb?kernel? in their name.
            \item
            \begin{cmd}
$ sudo apt install apache2-dev
         \end{cmd}
         \end{enumerate}

      \end{sol}

   \end{exe}

   \begin{exe} {Using APT to Find Information About a Package}

      Using \textbf{apt-cache} and \textbf{apt}
      (and not \textbf{dpkg}), find:

      \begin{enumerate}
         \item
         All packages that contain a reference to
         \textbf{bash} in their name or description.
         \item
         Installed and available \textbf{bash} packages.
         \item
         The package information for \textbf{bash}.
         \item
         The dependencies for the \textbf{bash} package.
      \end{enumerate}

      Try the commands you used above both as \texttt{root}
      and as a regular user. Do you notice any difference?

      \begin{sol}

         \begin{enumerate}
            \item
            \begin{cmd}
$ apt-cache search bash
         \end{cmd}
            \item
            \begin{cmd}
$ apt-cache search -n bash
         \end{cmd}
            \item
            \begin{cmd}
$ apt-cache show bash
         \end{cmd}
            \item
            \begin{cmd}
$ apt-cache depends bash
$ apt-cache rdepends bash
         \end{cmd}
         \end{enumerate}
      \end{sol}
   \end{exe}

   \begin{exe} {Managing Groups of Packages with \textbf{APT}}

      \textbf{APT} provides the ability to manage groups of
      packages, similarly to the way \textbf{dnf} does it,
      through the use of \textbf{metapackages}.  These can
      be thought of as \textbf{virtual packages}, that collect
      related packages that must be installed and
      removed as a group.

      To get a list of of available \textbf{metapackages}:
      \begin{cmd}
$ apt-cache search metapackage
   \end{cmd}

      \begin{out}[]
bacula - network backup service - metapackage
bacula-client - network backup service - client metapackage
bacula-server - network backup service - server metapackage
cloud-utils - metapackage for installation of upstream cloud-utils source
compiz - OpenGL window and compositing manager
emacs - GNU Emacs editor (metapackage)
....
   \end{out}
      You can then easily install them like regular single
      packages, as in:
      \begin{cmd}
$ sudo apt install bacula-client
   \end{cmd}

      \begin{out}[]
Reading package lists... Done
Building dependency tree
Reading state information... Done
The following extra packages will be installed:
  bacula-common bacula-console bacula-fd bacula-traymonitor
Suggested packages:
  bacula-doc kde gnome-desktop-environment
The following NEW packages will be installed:
  bacula-client bacula-common bacula-console bacula-fd bacula-traymonitor
0 upgraded, 5 newly installed, 0 to remove and 0 not upgraded.
Need to get 742 kB of archives.
After this operation, 1,965 kB of additional disk space will be used.
Do you want to continue? [Y/n]
   \end{out}
      Select an uninstalled metapackage and then remove it.
   \end{exe}

\end{Lab}











\section{APT}

\begin{frame}
   {What is APT?}

   \begin{itemize}
      \item Stands for \textbf{A}dvanced \textbf{P}ackaging
      \textbf{T}ool
      \item Includes utilities such as \textbf{apt} and
      \textbf{apt-cache} (which invoke the lower level
      \textbf{dpkg} program)
      \item Works with \textbf{Debian} packages (\texttt{.deb}
      file extension)
      \item Not uncommon to use a repository on more than one
      \textbf{Debian}-based \textbf{Linux} distribution.
   \end{itemize}
   \begin{info}[apt vs apt-get]
      For almost all interactive purposes, it is no longer necessary
      to use \textbf{apt-get}; one can just use the shorter
      name \textbf{apt}.  However, you will see \textbf{apt-get}
      used all the time out of habit, and it works better in
      scripts.  In this course we sometimes use \textbf{apt-get}
      because the commands we reference can be used either
      at the command line or in scripts.
   \end{info}

\end{frame}

\cprotect\note {

   For use on \textbf{Debian}-based systems, the
   \textbf{APT} (\textbf{A}dvanced \textbf{P}ackaging
   \textbf{T}ool) set of programs provides a higher level
   of intelligent services for using the underlying
   \textbf{dpkg} program, and plays the same role as
   \textbf{dnf} on \textbf{Red Hat}-based systems.  The
   main utilities are \textbf{apt} and
   \textbf{apt-cache}.  It can automatically resolve
   dependencies when installing, updating and removing
   packages.  It accesses external software
   \textbf{repositories}, synchronizing with them and
   retrieving and installing software as needed.

   Once again we are going to ignore graphical interfaces
   (on your computer) such as \textbf{synaptic} or the
   \textbf{Ubuntu Software Center}, or other older front
   ends to \textbf{APT} such as \textbf{aptitude}.


}

\section{APT Utilities}
\begin{frame}
   {apt, apt-get, apt-cache, etc.}

   \begin{itemize}
      \item Command-line tools for handling packages for
      \textbf{Debian Linux}
      \begin{itemize}
         \item Install/manage individual packages
         \item Upgrade packages
         \item Upgrade distribution
      \end{itemize}
      \item Front-ends to \textbf{dpkg}
      \item Also can retrieve packages from repositories
      \item Excellent resources for searching, examining and
      downloading packages:

      \url{https://www.debian.org/distrib/packages}

      \url{https://packages.ubuntu.com}

   \end{itemize}

\end{frame}

\cprotect\note{

   These utilities  are the \textbf{APT} command line
   tools for package management.   They can be used to install,
   manage and upgrade individual packages or the entire
   system, and can even upgrade the distribution to a
   completely new release, which can be a difficult task.

   There are even (imperfect) extensions that let
   \textbf{apt} work with \textbf{rpm} files.

   Like \textbf{dnf} and \textbf{zypper}, \textbf{APT}  works with
   multiple remote repositories.

}

\section{Queries}
\begin{frame}
   {Queries}

   \begin{itemize}
      \item \textbf{apt-cache}
      \begin{cmd}
$ apt-cache search
$ apt-cache show
$ apt-cache showpkg
$ apt-cache depends
      \end{cmd}
      \item \textbf{apt-file}
      \begin{cmd}
$ apt-file search
$ apt-file list
      \end{cmd}
      \item
      You may have to install \textbf{apt-file} first and
      update its database as in:
      \begin{cmd}
$ sudo apt install apt-file
$ sudo apt-file update
      \end{cmd}

   \end{itemize}


\end{frame}

\cprotect\note{

   \begin{cmd}
$ apt-cache search apache2
   \end{cmd}
   Search the repository for a package named \textbf{apache2}.

   \begin{cmd}
$ apt-cache show apache2
   \end{cmd}
   Display basic information about the \textbf{apache2}
   package.

   \begin{cmd}
$ apt-cache showpkg apache2
   \end{cmd}
   Display detailed information about the
   \textbf{apache2} package.

   \begin{cmd}
$ apt-cache depends apache2
   \end{cmd}
   List all dependent packages for \textbf{apache2}.

   \begin{cmd}
$ apt-file search apache2.conf
   \end{cmd}
   Search the repository for a file named
   \filelink{apache2.conf}.

   \begin{cmd}
$ apt-file list apache2
   \end{cmd}
   List all files in the \textbf{apache2} package.


}

\clearpage
\section{Installing/Removing/Upgrading Packages}
\begin{frame}
   {Installing/Removing/Upgrading Packages}
   \begin{itemize}
      \item
      Install and remove packages:
      \begin{cmd}
$ sudo apt install glibc
$ sudo apt remove glibc
$ sudo apt --purge remove glibc
      \end{cmd}

      \item
      \textbf{update} repository databases and then
      \textbf{upgrade} system or packages:
      \begin{cmd}
$ sudo apt update
$ sudo apt upgrade
$ sudo apt dist-upgrade
      \end{cmd}
      \begin{important}[No individual pakage uprades]
         Unlike with \textbf{dnf} or \textbf{zypper}, one
         never updates/upgrades individual packages; the entire
         package list is dealt with as a unit.
      \end{important}
   \end{itemize}
\end{frame}

\cprotect\note{



   \begin{cmd}
$ sudo apt install [package]
   \end{cmd}
   Used to install new packages or update a package which
   is already installed.

   \begin{cmd}
$ sudo apt remove [package]
   \end{cmd}
   Used to remove a package from the system. This does
   not remove the configuration files.

   \begin{cmd}
$ sudo apt --purge remove [package]
   \end{cmd}
   Used to remove a package and its configuration files
   from a system.
   \begin{cmd}
$ sudo apt update
   \end{cmd}
   Used to synchronize the package index files with
   their sources. The indexes of available packages are
   fetched from the location(s) specified in
   \filelink{/etc/apt/sources.list}.

   \begin{cmd}
$ sudo apt upgrade
$ sudo apt dist-upgrade
   \end{cmd}
   \verb?upgrade? is used to apply all available updates to
   packages already installed.  \verb?dist-upgrade? will not
   update to a whole new distribution version
   as is commonly misunderstood.

   Note that you must \textbf{update} before you
   \textbf{upgrade}, unlike with \textbf{dnf}
   or \textbf{zypper}, which can be confusing to habitual
   users of those utilities.

}

\section{Cleaning Up}
\begin{frame}
   {Cleaning Up}

   \begin{itemize}

      \item Get rid of any packages not needed anymore, such as
      older \textbf{Linux} kernel versions:
      \begin{cmd}
$ sudo apt autoremove
      \end{cmd}

      \item
      Clean out cache files and any archived package files
      that have been installed:
      \begin{cmd}
$ sudo apt clean
      \end{cmd}
      This can save a lot of space.

   \end{itemize}

\end{frame}

\cprotect\note {

   Using \textbf{apt clean} and \textbf{apt autoremove} is a quick and fairly safe way to free up disk space in
   \filelink{/var} when disk space is low.

}




\clearpage\section{Labs}\begin{Lab}

   \ifINSTRUCTORLED
      \begin{videobox}
         \begin{alltt}
           \reslink{using_swap_and_oom_demo.mp4}\end{alltt}
      \end{videobox}
   \fi

   \begin{exe} {Managing Swap Space}

      Examine your current swap space by doing:
      \begin{cmd}
$ cat /proc/swaps
   \end{cmd}

      \begin{out}[]
Filename                                Type            Size    Used    Priority
/dev/sda11                              partition       4193776 0       -1
   \end{out}
      We will now add more swap space by adding either a new
      partition or a file.  To use a file we can do:
      \begin{cmd}
$ dd if=/dev/zero of=swpfile bs=1M count=1024
   \end{cmd}

      \begin{out}[]
1024+0 records in
1024+0 records out
1073741824 bytes (1.1 GB) copied, 1.30576 s, 822 MB/s
   \end{out}
      \begin{cmd}
$ mkswap swpfile
   \end{cmd}

      \begin{out}[]
Setting up swapspace version 1, size = 1048572 KiB
no label, UUID=85bb62e5-84b0-4fdd-848b-4f8a289f0c4c
   \end{out}
      (For a real partition just feed \textbf{mkswap} the
      partition name, but be aware all data on it will be
      erased!)

      Activate the new swap space:
      \begin{cmd}
$ sudo swapon swpfile
   \end{cmd}

      \begin{out}[]
swapon: /tmp/swpfile: insecure permissions 0664, 0600 suggested.
swapon: /tmp/swpfile: insecure file owner 500, 0 (root) suggested.
   \end{out}
      \begin{redhatfamilybox}
         Notice \textbf{RHEL} warns us we are being insecure, we really
         should fix with:
         \begin{cmd}
$ sudo chown root:root swpfile
$ sudo chmod 600 swpfile
      \end{cmd}
      \end{redhatfamilybox}
      We ensure \verb?swpfile? is being used:
      \begin{cmd}
$ cat /proc/swaps
      \end{cmd}

      \begin{out}[]
Filename                                Type            Size    Used    Priority
/dev/sda11                              partition       4193776 0       -1
/tmp/swpfile                            file            1048572 0       -2
      \end{out}
      Note the \verb?Priority? field; swap partitions or files
      of lower priority will not be used until higher priority
      ones are filled.


      Remove the swap file from use and delete it to save
      space:
      \begin{cmd}
$ sudo swapoff swpfile
$ sudo rm swpfile
   \end{cmd}
   \end{exe}


   \begin{exe} {Invoking the \textbf{OOM} Killer}

      \begin{itemize}
         \item

         When the \textbf{Linux} kernel gets under extreme memory
         pressure it invokes the dreaded \textbf{OOM} (\textbf{O}ut
         \textbf{O}f \textbf{M}emory) \textbf{Killer}. This tries to
         select the ``best'' process to kill to help the
         system recover gracefully.

         \item

         We are going to force the system to run short on memory
         and watch what happens. The first thing to do is to open
         up a terminal window, and in it type:

         \begin{cmd}
$ sudo tail -f /var/log/messages
         \end{cmd}
         in order to watch kernel messages as they appear.

      \end{itemize}
      \begin{lfbox}
         \begin{itemize}
            \item
            An even better way to look is furnished by:
            \begin{cmd}
$ dmesg -w
            \end{cmd}
            as it does not show non-kernel messages.
         \end{itemize}
      \end{lfbox}
      \begin{itemize}

         \item

         This exercise will be easier to perform if we turn off
         all swap first with the command:

         \begin{cmd}
$ sudo /sbin/swapoff -a
         \end{cmd}
         Make sure you turn it back on later with
         \begin{cmd}
$ sudo /sbin/swapon -a
         \end{cmd}

         \item

         Now we are going to put the system under increasing
         memory pressure. You are welcome to find your own way of
         doing it but we also supply a program for consuming the
         memory:

         \cfile[\texttt{lab\_wastemem.c}]{lab_wastemem.c}

         It takes as an argument how many MB to consume.
         Keep running it, gradually increasing the amount of memory
         requested until your system runs out of memory.

         \begin{info}
            You should be able to compile the program and
            run it by just doing:
            \begin{cmd}
$ gcc -o lab_wastemem lab_wastemem.c
$ ./lab_wastemem 4096
            \end{cmd}
            which would waste 4 GB.  It would be a good idea
            to run \textbf{gnome-system-monitor} or another
            memory monitoring program while it is running
            (although the display may freeze for a while!)
         \end{info}

         \item

         You should see the \textbf{OOM} (Out of Memory) killer
         swoop in and try to kill processes in a struggle to stay
         alive. Who gets clobbered first?
      \end{itemize}

      \begin{sol}
         \begin{alltt}
\lablink{lab_wastemem.c}
\lablink{lab_waste.sh}
         \end{alltt}

      \end{sol}

   \end{exe}
\end{Lab}

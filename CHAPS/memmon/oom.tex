\section{Out of Memory Killer (OOM)}

\begin{frame}
   {OOM Killer}

   \begin{itemize}
      \item
      Systems may run out of physical memory.  One can:
      \begin{enumerate}
         \item Deny any further memory requests until memory is freed up
         \item Extend physical memory by the use of \textbf{swap} space
         \item Terminate (intelligently) selected processes to reduce memory usage
         and let the system survive
      \end{enumerate}
      \item \textbf{Linux} systems most often implement the second and third
      methods
      \item Which processes are terminated is selected by the
      \textbf{OOM} killer
   \end{itemize}
\end{frame}

\cprotect\note{

   The simplest way to deal with memory pressure would be to
   permit memory allocations to succeed as long as free memory
   is available and then fail when all memory is exhausted.

   The second simplest way is to use swap space on disk to push
   some of the resident memory out of core; in this case, the
   total available memory (at least in theory) is the actual
   RAM plus the size of the swap space. The hard part of this
   is to figure out which pages of memory to swap out when
   pressure demands. In this approach, once the swap space
   itself is filled, requests for new memory must fail.

   Linux, however, goes one better; it permits the system to
   overcommit memory, so that it can grant memory requests that
   exceed the size of RAM plus swap. While this might seem
   foolhardy, many (if not most) processes do not use all
   requested memory.

   An example would be a program that allocates a 1 MB buffer,
   and then uses only a few pages of the memory. Another
   example is that every time a child process is forked, it
   receives a copy of the entire memory space of the
   parent. Because Linux uses the COW (copy on write)
   technique, unless one of the processes modifies memory, no
   actual copy needs be made. However, the kernel has to assume
   that the copy might need to be done.

   Thus, the kernel permits overcommission of memory, but only
   for pages dedicated to user processes; pages used within the
   kernel are not swappable and are always allocated at request
   time.


}

\begin{frame}
   {OOM Killer Algorithms}

   \begin{itemize}

      \item
      Heuristic algorithm is not intended to be depended on for
      normal operations.  Is there more for a graceful shutdown or
      retrenchment
      \item
      Process selection depends on a \textbf{badness} value which can
      be read from \filelink{/proc/[pid]/oom_score} for each process
      \item
      Adjustments can be made to a process's \file{oom\_adj\_score} in the
      same directory for each task.
   \end{itemize}

\end{frame}

\cprotect\note{

   One can modify, and even turn off overcommission by
   setting the value of
   \filelink{/proc/sys/vm/overcommit_memory}:
   \begin{itemize}

      \item 0: (default) Permit overcommission, but refuse
      obvious overcommits, and give root users somewhat more
      memory allocation than normal users.

      \item 1: All memory requests are allowed to overcommit.

      \item 2: Turn off overcommission. Memory requests will fail
      when the total memory commit reaches the size of the
      swap space plus a configurable percentage (50 by
      default) of RAM. This factor is modified changing
      \filelink{/proc/sys/vm/overcommit_ratio}.

   \end{itemize}
   An amusing take on this was given by Andries Brouwer
   (\url{https://lwn.net/Articles/104185/}):

   \begin{quote}\small{

         An aircraft company discovered that it was cheaper to
         fly its planes with less fuel on board. The planes
         would be lighter and use less fuel and money was saved.
         On rare occasions however the amount of fuel was
         insufficient, and the plane would crash. This problem
         was solved by the engineers of the company by the
         development of a special OOF (out-of-fuel) mechanism.
         In emergency cases a passenger was selected and thrown
         out of the plane. (When necessary, the procedure was
         repeated.) A large body of theory was developed and
         many publications were devoted to the problem of
         properly selecting the victim to be ejected. Should the
         victim be chosen at random? Or should one choose the
         heaviest person? Or the oldest? Should passengers pay
         in order not to be ejected, so that the victim would be
         the poorest on board? And if for example the heaviest
         person was chosen, should there be a special exception
         in case that was the pilot? Should first class
         passengers be exempted? Now that the OOF mechanism
         existed, it would be activated every now and then, and
         eject passengers even when there was no fuel shortage.
         The engineers are still studying precisely how this
         malfunction is caused.
      }
   \end{quote}



}

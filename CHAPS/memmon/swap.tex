\clearpage
\section{Swap}
\begin{frame}
   {Using swap}

   \begin{itemize}
      \item Virtual memory on disk
      \item Create swap partition in \textbf{fdisk}
      \item \textbf{mkswap}: format swap partitions or files
      \item \textbf{swapon}: activate swap partitions or files
      \item \textbf{swapoff}: deactivate swap partitions or files
      \item Can mount at boot time (using
      \filelink{/etc/fstab})
      \item Can have more than one swap partition
   \end{itemize}
   \begin{cmd}
$ cat /proc/swaps
   \end{cmd}

   \begin{out}[]
Filename           Type            Size    Used    Priority
/dev/sda6          partition       8290300 0       -1
/tmp/swpfile       file            102396  0       -2
   \end{out}

\end{frame}

\cprotect\note{

   \textbf{Linux} employs a \textbf{virtual memory} system
   in which the operating system can function as if it had
   more memory than it really does. This kind of memory
   \textbf{overcommission} functions in two ways:
   \begin{itemize}
      \item

      Many programs do not actually use all the memory they
      are given permission to use. Sometimes this is
      because child processes inherit a copy of the
      parent's memory regions utilizing a \textbf{COW}
      (\textbf{C}opy \textbf{O}n \textbf{W}rite) technique
      in which the child only obtains a unique copy (on a
      page by page basis) when there is a change.

      \item When memory pressure becomes important less
      active memory regions may be \textbf{swapped} out to
      disk, to be recalled only when needed again.
   \end{itemize}

   Such swapping is usually done to one or more dedicated
   partitions or files; \textbf{Linux} permits multiple
   \textbf{swap areas} so the needs can be adjusted
   dynamically. Each area has a priority and lower
   priority areas are not used until higher priority areas
   are filled.

   In most situations the recommended swap size is the
   total \textbf{RAM} on the system. You can see what your
   system is currently using for swap areas with
   \texttt{cat /proc/swaps} and report on current usage
   with \texttt{free}.  At any given time most memory is
   in use for caching file contents to prevent actually
   going to the disk any more than necessary, or in a
   sub-optimal order or timing. Such pages of memory are
   never swapped out as the backing store is the files
   themselves, so writing out to swap would be pointless;
   instead \textbf{dirty} pages (memory containing updated
   file contents that no longer reflect the stored data)
   are flushed out to disk.

   It is also worth pointing out that in \textbf{Linux}
   memory used by the kernel itself, as opposed to
   application memory, is \textbf{never swapped} out, in
   distinction to some other operating systems.

}




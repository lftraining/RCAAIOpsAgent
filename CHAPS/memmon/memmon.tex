\section{Memory Monitoring and Tuning}
\begin{frame}
   {Memory Monitoring}

   Important tools for monitoring and tuning memory in \textbf{Linux}

   \lftable{Memory Monitoring Utilities}

   \small{\begin{tcolorbox}[tabularx={p{0.25\textwidth}|X|p{0.25\textwidth}}]

         \textbf{Utility}
         &
         \textbf{Purpose}
         &
         \textbf{Package}
         \\ \hline
         \textbf{free}
         &
         Brief summary of memory usage
         &
         procps
         \\ \hline
         \textbf{vmstat}
         &
         Detailed virtual memory statistics and block I/O,
         dynamically updated
         &
         procps
         \\ \hline
         \textbf{pmap}
         &
         Process memory map
         &
         procps
      \end{tcolorbox}}

   \begin{cmd}
$ free -m
   \end{cmd}

   \begin{out}[]
           total     used     free      shared  buff/cache   available
Mem:       15901     2080     9317         992        4502       12457
Swap:       8095        0     8095
   \end{out}


\end{frame}

\cprotect\note{

   Over time, systems have become more demanding of
   memory resources at the same time RAM prices have
   decreased and performance has improved.

   Yet, it is often the case that bottlenecks in
   overall system performance and throughput are
   memory-related; the CPUs and the I/O subsystem can
   be waiting for data to be retrieved from or written
   to memory.

   There are many tools for monitoring, debugging and tuning
   a system's behavior with regard to its memory.

   Tuning the memory sub-system can be a complex
   process. First of all, one has to take note that memory
   usage and I/O throughput are intrinsically related, as, in
   most cases, most memory is being used to cache the
   contents of files on disk.

   Thus, changing memory parameters can have a large effect
   on I/O performance, and changing I/O parameters can have
   an equally large converse effect on the virtual memory
   sub-system.

}
\clearpage
\begin{frame}
   {/proc/meminfo}

   \begin{cmd}
$ cat /proc/meminfo
   \end{cmd}

   %%   \begin{multicols}{3}
   \respfile{meminfo.output}
   %%   \end{multicols}
\end{frame}

\cprotect\note{

   The pseudo file \filelink{/proc/meminfo} contains a wealth of
   information about how memory is being used.

}
\section{/proc/sys/vm}

\begin{frame}
   {/proc/sys/vm}

   \begin{itemize}

      \item
      Contains many tunable knobs to
      control the Virtual Memory system
      \item
      Almost all of the entries are writable (by root)

      \item
      Exactly what appears in
      this directory will depend somewhat on the kernel
      version.

      \item
      Values can be changed either by directly
      writing to the entry, or using \textbf{sysctl}

   \end{itemize}

\end{frame}

\cprotect\note{

   When tweaking parameters in \filelink{/proc/sys/vm}, the
   usual best practice is to adjust one thing at a time and
   look for effects. The primary (inter-related) tasks are:

   \begin{itemize}
      \item
      Controlling flushing parameters; i.e., how many pages
      are allowed to be dirty and how often they are flushed
      out to disk.
      \item
      Controlling swap behavior; i.e., how much pages that
      reflect file contents are allowed to remain in memory,
      as opposed to those that need to be swapped out as they
      have no other backing store.
      \item
      Controlling how much memory overcommission is allowed,
      since many programs never need the full amount of memory
      they request, particularly because of copy on write
      (COW) techniques.
   \end{itemize}
   Memory tuning can be subtle: what works in one
   system situation or load may be far from optimal in other
   circumstances.

   \begin{figure}[H]
      \includegraphics[width=5.25in]{IMAGES/procsysvm}
      \caption{/proc/sys/vm}
   \end{figure}

}
\section{vmstat}
\begin{frame}
   {vmstat}

   \begin{itemize}

      \item
      Displays information about memory, paging, I/O, processor
      activity and processes.

      \item
      Many options
      \begin{cmd}
$ vmstat [options] [delay] [count]
      \end{cmd}

      \item
      \verb?delay? is given in seconds, report is repeated \verb?count? times.

      \item
      First line shows averages since the last
      reboot
      \item
      Succeeding lines show activity during the
      specified interval.
   \end{itemize}

   \begin{figure}[H]
      \includegraphics[width=6in]{IMAGES/vmstat}
      \caption{vmstat}
   \end{figure}

\end{frame}

\cprotect\note{

   If the option \verb?-S m? is given, memory statistics will
   be in MB instead of KB.

   With the \verb?-a? option, \textbf{vmstat} displays
   information about active and inactive memory pages:
   active pages are those recently used; they may be clean
   (disk contents are up to date) or dirty (need to be
   flushed to disk eventually). By contrast, inactive memory
   pages have not been recently used and are more likely to
   be clean and are released sooner under memory pressure:

   Memory moves back and forth between active and inactive
   lists, with new references, or a long time
   between uses.

   \begin{figure}[H]
      \includegraphics[width=4.9in]{IMAGES/vmstata}
      \caption{Using vmstat}
   \end{figure}

   \vspace{-24pt}

   If you just want to get some quick statistics on only one
   partition, use the \verb?-p? option

   \begin{figure}[H]
      \includegraphics[width=4.9in]{IMAGES/vmstatp}
      \caption{Using vmstat on One Disk}
   \end{figure}
   \vspace{-24pt}

}








